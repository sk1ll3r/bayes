\subsection{Dirichlet process mixtures}
The purpose is to cluster observations. We can't model continuous observations directly using a Dirichlet processes because the samples from them are almost surely discrete probability measures. Also, the posterior measure assigned to $x_i$ would never beinfluenced by observations $x_j \neq x_i$, regardless of their proximity.

The Dirichlet process mixtures model is as follows:
\begin{align*}
	G 				&\sim \DP(\alpha, H) \\
	\bar \theta_n	&\sim G 				& n = 1, \dotsc, N \\
	x_n				&\sim F(\bar \theta_n)
\end{align*}
where $G$ is being sampled from $\DP(\alpha, H)$ via the stick-breaking construction:
\begin{align*}
	\vec \pi 	&\sim \GEM(\alpha)	& \vec \pi = (\pi_1, \pi_2, \dotsc)\\
	\theta_k	&\sim H(\lambda)	& k = 1, 2, \dotsc \\
	G(\theta)	&= \sum_{k = 1}^\infty \pi_k \delta(\theta, \theta_k)
\end{align*}
this solves the problem of inability of the $\DP$ to model the distribution of observations directly. Now two observations $x_i, x_j$ are considered to be from the same cluster of $\bar \theta_n$ if both are $\sim F(\bar \theta_n)$.