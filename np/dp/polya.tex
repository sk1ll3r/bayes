\subsection{Pólya urn construction}
The purpose is to generate samples from the posterior predictive, $p(\tilde x \mid \mathcal D, \alpha, H)$ where $\mathcal D = \{x_n: x_n \sim G, G \sim DP(\alpha, H)\}$.

\begin{theorem}
	Let $G \sim \DP(\alpha, H)$. Let $h(x)$ be the density of the base measure $H$. Consider a set of $N$ observations $x_n \sim G$ taking $K \leq N$ distinct values $\{\bar x_k\}_{k = 1}^K$. The predictive distribution of the next observation is
	\begin{equation}
		p(\tilde x \mid x_1, \dotsc, x_N, \alpha, H) = \frac{1}{\alpha + N} \left(\alpha h(\tilde x) + \sum_k N_k \delta(\tilde x, \bar x_k)\right) \label{eqn:np/dp/polya/post-pred}
	\end{equation}
	where $N_k = \sum_n \delta(x_n, \bar x_k)$ is the count of observations that equal $\bar x_k$.
\end{theorem}

\begin{proof}
	TODO
\end{proof}

We can get a sample from this distribution via the generalised Pólya urn model:
	\begin{algorithmbis}[Pólya urn construction]\label{alg:np/dp/polya/polya}
        \begin{algorithmic}[1]
            \State Assume we have a bag with $N$ identical balls of $K$ different colours (our observations) with the probability of drawing each of them being $\frac{1}{\alpha + N}$. We also have a special black ball which can be drawn with probability $\frac{\alpha}{\alpha + N}$.
            \State Draw a ball.
            \If{it's not black}
            	\State Record colour.
            	\State (Put back and add one more ball with the same colour.)
            \Else
            	\State Draw a ball from the bag of yet unseen colours, following $H$.
            	\State Record new colour.
            	\State (Put back both the black ball and the ball with a new colour.)
            \EndIf
            \State The recorded colour follows the posterior predictive in \eqref{eqn:np/dp/polya/post-pred}.
        \end{algorithmic}
    \end{algorithmbis}
This follows \eqref{eqn:np/dp/polya/post-pred} exactly.