\section{How it works}
\subsection{Notation}
The syntax is as follows
\begin{verbatim}
	[assume symbol <expr>]
	[observe (<random proc> <arg> ... <arg>) <const>]
	[predict <expr>]
\end{verbatim}
where \verb!assume!'s are either deterministic or random variables declarations, \verb!observe!'s condition the distribution of the \verb!assume!'d variables and \verb!predict!'s give samples from the posteriors of the corresponding \verb!<expr>!'s.

Probability of an execution trace is
\begin{align}
	\tilde p(\vec y, \vec x) 				&= \prod_{n = 1}^N p(y_n \mid \vec \theta_{t_n}, \vec x_n) \tilde p(\vec x_n \mid \vec x_{n - 1}) \\
	\tilde p(\vec x_n \mid \vec x_{n - 1}) 	&= \prod_{k = 1}^{\vec x_n \setminus \vec x_{n - 1}} p(x_{n, k} \mid \vec \theta_{t_{n, k}}, x_{n, 1:(k - 1)}, \vec x_{n - 1}) \\
	p(y_n \mid \vec \theta_{t_n}, \vec x_n)	&= \text{likelihood of observed output } y_n \\
	\text{tilde}							&= \text{distributions we can only sample from} \\
	y_n										&= n^{\text{th}} \text{ }\mathtt{observe}\text{'d output} \\
	t_n										&= \text{type of } n^{\text{th}} \text{ }\mathtt{observe}\text{'d main }\mathtt{random\ proc} \\
	\vec \theta_{t_n}						&= \text{arguments of } t_n \\
	\vec x_n								&= \text{set of all random procedure application results computed } \nonumber\\
											&\text{before }p(y_n \mid \vec \theta_{t_n}, \vec x_n) \text{ is evaluated. I.e. before the } n^{\text{th}} \text{ observe.}
\end{align}

Whevever a \verb!predict! is called, we want to sample from $\tilde p(\vec x \mid \vec y) \propto \tilde p(\vec y, \vec x)$. A general overview of this can be seen in Figure~\ref{fig:pprog/how/general}.
\begin{figure}[!htb]
\centering
\includegraphics[scale=1]{pprog/how/figures/general/general}
\caption{A general overview of Anglican interpretation.}
\label{fig:pprog/how/general}
\end{figure}

\subsection{Random databse}
This is an Metropolis-Hastings (see Subsection~\ref{subsec:sampling/mcmc-mh/mh}) approach to inference. The proposal step of the MH algorithm is illustrated in Figure~\ref{fig:pprog/how/figures/rdb}. 
\begin{figure}[!htb]
\centering
\includegraphics[scale=1]{pprog/how/figures/rdb/rdb}
\caption{Illustration of the RDB proposal step.}
\label{fig:pprog/how/figures/rdb}
\end{figure}
Following the Algorithm~\ref{alg:sampling/mcmc-mh/mh}, the proposal step consists of these steps:
\begin{itemize}
	\item Pick a single variable $x_{n, k}$ from the $|\vec x|$ random draws uniformly randomly.
	\item Get a new random choice $x'_{n, k}$ by sampling from a kernel $x'_{n, k} \sim \kappa(\cdot \mid x_{n, k})$. 
	\item Continue interpretation of program to get a new set of variables, $\vec x'$, that correspond to a new valid execution trace. (whenever a random procedure in the interpretation is the same as in $\vec x$, we reuse the existing value, only rescoring the conditional probability when necessary).
	\item $\vec x'$ is our MH proposal.
\end{itemize}
Following this procedure and notation in Figure~\ref{fig:pprog/how/figures/rdb}, the proposal distribution can be expressed as
\begin{equation}
	q(\vec x' \mid \vec x)	= \frac{\kappa(x'_{n, k} \mid x_{n, k})}{|\vec x|} \frac{p(\vec x' \setminus \vec x \mid \vec x' \cap \vec x)}{p(x'_{n, k} \mid \vec x' \cap \vec x)}
\end{equation}
The $1 / {|\vec x|}$ corresponds to randomly uniformly choosing a single variable. The $\kappa(x'_{n, k} \mid x_{n, k})$ corresponds to the proposal kernel. And finally,
\begin{align*}
	\frac{p(\vec x' \setminus \vec x \mid \vec x' \cap \vec x)}{p(x'_{n, k} \mid \vec x' \cap \vec x)}	&= p\left(\{\vec x' \setminus \vec x\} \setminus x'_{n, k} \mid x'_{n, k}, \vec x \cap \vec x'\right) \\
																										&= p\left(\{\vec x' \setminus \vec x\} \setminus x'_{n, k} \mid \{\vec x \cap \vec x'\} \cup \{x'_{n, k}\}\right) \\
\end{align*}
which corresponds to the probability of ``the rest of execution given the past random choices of this proposed execution trace''.

The acceptance probability can be written as
\begin{align}
	\mathcal A(\vec x' \mid \vec x)	&= \min\left(1, \frac{p(\vec y', \vec x') q(\vec y, \vec x \mid \vec y', \vec x')}{p(\vec y, \vec x) q(\vec y', \vec x' \mid \vec y, \vec x)}\right) \nonumber\\
									&= \min\left(1, \frac{p(\vec y \mid \vec x') p(\vec x') q(\vec x \mid \vec x')}{p(\vec y \mid \vec x) p(\vec x) q(\vec x' \mid \vec x)}\right) \label{eqn:pprog/how/rdb/acc}
\end{align}

In the \emph{propose from prior} case, a new random choice $x'_{n, k}$ is obtained by continuing the interpretation from $x_{n, k - 1}$ which means the expression of the kernel becomes $\kappa(x'_{n, k} \mid x_{n, k}) = p(x'_{n, k} \mid \vec x' \cap \vec x)$. Note that the reverse kernel becomes $\kappa(x_{n, k} \mid x'_{n, k}) = p(x_{n, k} \mid \vec x \cap \vec x')$. Hence the expressions for the proposal distribution (in both ways) become
\begin{align*}
	q(\vec x' \mid \vec x)	&= \frac{p(\vec x' \setminus \vec x \mid \vec x' \cap \vec x)}{|\vec x|} \\
	q(\vec x \mid \vec x')	&= \frac{p(\vec x \setminus \vec x' \mid \vec x \cap \vec x')}{|\vec x'|}
\end{align*}
Substituting this to the acceptance probability in \eqref{eqn:pprog/how/rdb/acc}, we obtain
\begin{equation}
	\mathcal A(\vec x' \mid \vec x) = \min\left(1, \frac{p(\vec y \mid \vec x') p(\vec x') |\vec x| p(\vec x \setminus \vec x' \mid \vec x \cap \vec x')}{p(\vec y \mid \vec x) p(\vec x) |\vec x'| p(\vec x' \setminus \vec x \mid \vec x' \cap \vec x)}\right)
\end{equation}

Summary: we just keep proposing and accepting and whenever a \verb!predict! is needed, we just report the current (or the corresponding function of a subset of) $\vec x$.
\subsection{Sequential Monte Carlo}
In this case, we follow the Algorithm~\ref{alg:sampling/part-fil/part-fil}. We show in Figure~\ref{fig:pprog/how/figures/smc} an illustration for one SMC iteration, adopting the notation in this chapter.
\begin{figure}[!htb]
\centering
\includegraphics[scale=0.75]{pprog/how/figures/smc/smc}
\caption{Illustration of the SMC iteration $n$.}
\label{fig:pprog/how/figures/smc}
\end{figure}

\begin{itemize}
	\item The proposal is done by just continuing interpretation, i.e. $q\left(\vec x_n^{(\ell)} \mid \vec x_{n - 1}^{A_{n - 1}^{(\ell)}}, \vec y_n; \vec \theta\right) = p\left(\vec x_n^{(\ell)} \mid \vec x_{n - 1}^{A_{n - 1}^{(\ell)}} \right)$.
	\item The weights calculation then simplifies to $p\left(y_n \mid \vec x_{n - 1}^{A_{n - 1}^{(\ell)}}\right)$. TODO: Verify.
	\item Whenever a \verb!predict! is needed, we can resample from
		\begin{equation*}
			\hat p(\mathrm d \vec x_{1:n} \mid y_{1:n}; \vec \theta) = \sum_\ell \hat w_t^{(\ell)} \delta_{\vec x_{1:n}^{(\ell)}}(\mathrm d \vec x_{1:n})
		\end{equation*}
		to get a sample from the posterior.
\end{itemize}
\subsection{Particle Gibbs}
Here, we follow the Particle Gibbs algorithm described in Algorithm~\ref{alg:sampling/pmcmc-pg-pg/pg}. The illustrations are below in Figure~\ref{fig:pprog/how/figures/pgibbs1}, and Figure~\ref{fig:pprog/how/figures/pgibbs2}.
\begin{figure}[!htb]
\centering
\includegraphics[scale=0.75]{pprog/how/figures/pgibbs/pgibbs1}
\caption{Initialisation of the Particle Gibbs sampler.}
\label{fig:pprog/how/figures/pgibbs1}
\end{figure}

\begin{figure}[!htb]
\centering
\includegraphics[scale=0.8]{pprog/how/figures/pgibbs/pgibbs2}
\caption{Sweep $s$ of the Particle Gibbs sampler.}
\label{fig:pprog/how/figures/pgibbs2}
\end{figure}